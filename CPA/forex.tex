
\begin{itemize}

\item \textit{风险控制}:止损位永远设在图表上重要的价位之外,宁可减少交易量去迁就一个安全的止损位,应避免参与行情过于激烈的品种!
\item \textit{操作周期}:要在4小时以上,最好是日图和周图!
\item 感情上应该远离市场,不要有妄想症和强迫症,不是非要开仓不可!
\item 不管多小的头寸都应该是经过慎重考虑的!
\item 每个月的回测不能超过10\%,这里包括赚的钱!
\item 在什么周期上开的仓位,就在什么周期上平仓,4小时,至少持有2-5天;日图上至少持有2-3周;周图上至少持有1-2个月!由于周期越长,风险越大,因此头寸也应该越小!
\item 钱要到一定数量级才才能显现出其威力,所以将积累钱财做疯狂的兴趣!
\item \textbf{成功的关键就是长期持续的掌握优势}!
\end{itemize}



\section*{识别趋势的改变}


\subsection*{长期盘整}

对于长期盘整而言,这是最有效的指标,不论是哪个方向,长期盘整的突破,都是有钱可赚的!

\subsection*{123法则}

\begin{itemize}
\item 突破趋势线
\item 测试的高点或低点
\item 向下跌破前一波回档的低点或者向上穿越前一波反弹的高点
\end{itemize}

\textbf{注意避免挨耳光:}

1.交易仅选择流动性高,且历史上甚少出现突然而大幅的反转走势的市场。由图形中观察,流动性欠佳的市场经常出现成交量不大而价格剧烈波动的走势。

2.在可能的范围内,避开对于消息面非常敏感的市场, 或对于政府货币与财政政策反应剧烈的市场。由图形中观察,这类市场经常出现“跳空缺口”(gaps)—— 价格出现突然而巨幅的变动,而中间的价格并未显示

3.唯有当你可以将出场点设定在先前的支撑或压力价位, 才可以建立交易头寸。这种情况下,如果市场证明你的判断错误,你可以迅速认赔。


1-2-3准则是一种简单而有效的交易方法,如果谨慎运用,成功的机会远大十失败。然而,它有一项缺点,三个条件完全满足时,你通常已经错失一段相当大的行情,所以,此处将讨论一种可以协助你提早建立头寸的法则,它是我个人偏爱的方法,我称它为2B准则。


\subsection*{2B法则}
在上升趋势中,如果价格已经创新高而未能持续挺升,稍后又跌破先前的高点,则趋势很可能会发生反转。下降趋势也是如此。这项法则适用于三种趋势中的任何一种:短期、中期与长期.


依据2B准则交易,如果市场出现不利的走势,你必须当机立断,即刻承认错误。例如,在当日冲销的交易中,如果你根据2B卖空,当价格再度回升穿越新髙价时,你必须立即平仓出场。

如果价格仍然未能正式突破高价,你可以再卖空,但必须限制自己的损失,并让自己捱耳光出场。只要能够迅速认赔, 你便可以“留得青山在,•不怕没柴烧”。就当日冲销来说,2B准则的成功机会大约仅有50\%,错误时,如果可以限制损失, 正确时,让获利头寸持续发展,你可以因此享有巨额的获利。 就中期与长期来说,2B准则的成功机会则远较为高。


重点是:从事交易时,你没有必要知道每一项可以被知道的知识。事实上,你不可能知道所有的相关知识;如果你根据特定市场的知识交易,则你很可能因为一件你不知道或未曾考虑的事件而失败。换言之,失败可能来自于分析报告未提供的资料。

