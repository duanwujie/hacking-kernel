\chapter{文件系统}

\section{EXT2}

\subsection{定义}
\subsection{硬盘组织结构}
\subsection{目录结构}


\textbf{目标:能够自己写一个文件系统。}

VFS使用面向对象的设计思路,VFS中有4个主要的对象类型:

\begin{itemize}
\item 超级块对象(super\_block):它表示一个具体的已安装的文件系统
\item 索引节点对象(inode):它表示一个具体的文件
\item 目录项对象(dentry):它表示一个目录项,是路径的一个组成部分。
\item 文件对象(file):它表示进程打开的文件。
\end{itemize}

VFS将目录当作文件来处理,所以不存在目录对象,目录项代表的是路径中的一个组成部分。



\section{file\_system\_type}

	内核使用该结构体来描述文件系统的功能和行为:

\begin{lstlisting}[language={[ANSI]C},
        basicstyle=\tiny\ttfamily,
        stringstyle=\color{purple},
        keywordstyle=\color{blue}\bfseries,
        commentstyle=\color{olive},
        directivestyle=\color{blue},
        frame=shadowbox,
        %framerule=0pt,
        %backgroundcolor=\color{pink},
        rulesepcolor=\color{red!20!green!20!blue!20}
        %rulesepcolor=\color{brown}
        %xleftmargin=2em,xrightmargin=2em,aboveskip=0em
        ]
        
struct file_system_type {
	const char *name;/* 文件系统名字 */
	int fs_flags;	/* 文件系统类型标志 */
#define FS_REQUIRES_DEV		1 
#define FS_BINARY_MOUNTDATA	2
#define FS_HAS_SUBTYPE		4
#define FS_USERNS_MOUNT		8	/* Can be mounted by userns root */
#define FS_USERNS_DEV_MOUNT	16 /* A userns mount does not imply MNT_NODEV */
#define FS_USERNS_VISIBLE	32	/* FS must already be visible */
#define FS_RENAME_DOES_D_MOVE	32768	/* FS will handle d_move() during rename() internally. */
	struct dentry *(*mount) (struct file_system_type *, int,
		       const char *, void *);
	void (*kill_sb) (struct super_block *);/* 用来终止访问 super_block */
	struct module *owner;/* 文件系统模块 */
	struct file_system_type * next;
	struct hlist_head fs_supers;/* supperblock list */

	/* Runtime lock */
	struct lock_class_key s_lock_key;
	struct lock_class_key s_umount_key;
	struct lock_class_key s_vfs_rename_key;
	struct lock_class_key s_writers_key[SB_FREEZE_LEVELS];

	struct lock_class_key i_lock_key;
	struct lock_class_key i_mutex_key;
	struct lock_class_key i_mutex_dir_key;
};
        
        
\end{lstlisting}

\section{inode}

内核处理文件的关键是inode,每个文件(和目录)都有且只有一个对应的inode,其中包含元数据(如访问权限,上次
修改的日期,等等)和指向文件数据的指针。

\begin{lstlisting}[language={[ANSI]C},
        basicstyle=\tiny\ttfamily,
        stringstyle=\color{purple},
        keywordstyle=\color{blue}\bfseries,
        commentstyle=\color{olive},
        directivestyle=\color{blue},
        frame=shadowbox,
        %framerule=0pt,
        %backgroundcolor=\color{pink},
        rulesepcolor=\color{red!20!green!20!blue!20}
        %rulesepcolor=\color{brown}
        %xleftmargin=0em,xrightmargin=0em,aboveskip=0em
        ]


/*
 * Keep mostly read-only and often accessed (especially for
 * the RCU path lookup and 'stat' data) fields at the beginning
 * of the 'struct inode'
 */
struct inode {/* fs.h */
	umode_t			i_mode;/* 文件访问权限和所有权 */
	unsigned short		i_opflags;
	kuid_t			i_uid;/* uid about the file */
	kgid_t			i_gid;/* gid about the file */
	unsigned int		i_flags;

#ifdef CONFIG_FS_POSIX_ACL
	struct posix_acl	*i_acl;
	struct posix_acl	*i_default_acl;
#endif
	/* 负责管理结构性操作(如删除一个文件)和文件相关的元数据(例如属性) */
	const struct inode_operations	*i_op;
	struct super_block	*i_sb;
	struct address_space	*i_mapping;

#ifdef CONFIG_SECURITY
	void			*i_security;
#endif

	/* Stat data, not accessed from path walking */
	/* 对给定的文件系统,唯一的编号标识 */
	unsigned long		i_ino;
	/*
	 * Filesystems may only read i_nlink directly.  They shall use the
	 * following functions for modification:
	 *
	 *    (set|clear|inc|drop)_nlink
	 *    inode_(inc|dec)_link_count
	 */
	union {
		/* 记录使用该 inode 的硬链接总数 */
		const unsigned int i_nlink;
		unsigned int __i_nlink;
	};
	dev_t			i_rdev;
	
	loff_t			i_size;/* 文件大小 */
	struct timespec		i_atime;/* 最后访问时间 */
	struct timespec		i_mtime;/* 最后修改时间*/
	struct timespec		i_ctime;/* inode 最后修改时间 */
	spinlock_t		i_lock;	/* i_blocks, i_bytes, maybe i_size */
	unsigned short          i_bytes;
	unsigned int		i_blkbits;
	blkcnt_t		i_blocks;/*指定了按块存放的长度*/

#ifdef __NEED_I_SIZE_ORDERED
	seqcount_t		i_size_seqcount;
#endif

	/* Misc */
	unsigned long		i_state;
	struct mutex		i_mutex;

	unsigned long		dirtied_when;	/* jiffies of first dirtying */
	unsigned long		dirtied_time_when;

	struct hlist_node	i_hash;
	struct list_head	i_io_list;	/* backing dev IO list */
#ifdef CONFIG_CGROUP_WRITEBACK
	struct bdi_writeback	*i_wb;		/* the associated cgroup wb */

	/* foreign inode detection, see wbc_detach_inode() */
	int			i_wb_frn_winner;
	u16			i_wb_frn_avg_time;
	u16			i_wb_frn_history;
#endif
	struct list_head	i_lru;		/* inode LRU list */
	struct list_head	i_sb_list;
	union {
		struct hlist_head	i_dentry;
		struct rcu_head		i_rcu;
	};
	u64			i_version;
	atomic_t		i_count;/* 访问该inode的进程数目 */
	atomic_t		i_dio_count;
	atomic_t		i_writecount;
#ifdef CONFIG_IMA
	atomic_t		i_readcount; /* struct files open RO */
#endif

	const struct file_operations	*i_fop;	/* 用于操作文件中包含的数据 */
	struct file_lock_context	*i_flctx;
	struct address_space	i_data;
	struct list_head	i_devices;
	union {
		struct pipe_inode_info	*i_pipe;
		struct block_device	*i_bdev;
		struct cdev		*i_cdev;
		char			*i_link;
	};

	__u32			i_generation;

#ifdef CONFIG_FSNOTIFY
	__u32			i_fsnotify_mask; /* all events this inode cares about */
	struct hlist_head	i_fsnotify_marks;
#endif

	void			*i_private; /* fs or device private pointer */
};


\end{lstlisting}


\section{inode\_operations}

大多数请况下,各个函数指针成员的意义可以根据其名称推断。它们与对应的系统调用和用户空间工具
在名称方面非常相似。

\begin{lstlisting}[language={[ANSI]C},
        basicstyle=\tiny\ttfamily,
        stringstyle=\color{purple},
        keywordstyle=\color{blue}\bfseries,
        commentstyle=\color{olive},
        directivestyle=\color{blue},
        frame=shadowbox,
        %framerule=0pt,
        %backgroundcolor=\color{pink},
        rulesepcolor=\color{red!20!green!20!blue!20}
        %rulesepcolor=\color{brown}
        %xleftmargin=2em,xrightmargin=2em,aboveskip=0em
        ]
struct inode_operations {
	/* lookup 根据文件系统对象的名称(表示为字符串)查找其 inode 实例*/
	struct dentry * (*lookup) (struct inode *,struct dentry *, unsigned int);
	const char * (*follow_link) (struct dentry *, void **);
	int (*permission) (struct inode *, int);
	struct posix_acl * (*get_acl)(struct inode *, int);

	int (*readlink) (struct dentry *, char __user *,int);
	void (*put_link) (struct inode *, void *);

	int (*create) (struct inode *,struct dentry *, umode_t, bool);
	int (*link) (struct dentry *,struct inode *,struct dentry *);
	int (*unlink) (struct inode *,struct dentry *);
	int (*symlink) (struct inode *,struct dentry *,const char *);
	int (*mkdir) (struct inode *,struct dentry *,umode_t);
	int (*rmdir) (struct inode *,struct dentry *);
	int (*mknod) (struct inode *,struct dentry *,umode_t,dev_t);
	int (*rename) (struct inode *, struct dentry *,
			struct inode *, struct dentry *);
	int (*rename2) (struct inode *, struct dentry *,
			struct inode *, struct dentry *, unsigned int);
	int (*setattr) (struct dentry *, struct iattr *);
	int (*getattr) (struct vfsmount *mnt, struct dentry *, struct kstat *);
	int (*setxattr) (struct dentry *, const char *,const void *,size_t,int);
	ssize_t (*getxattr) (struct dentry *, const char *, void *, size_t);
	ssize_t (*listxattr) (struct dentry *, char *, size_t);
	int (*removexattr) (struct dentry *, const char *);
	int (*fiemap)(struct inode *, struct fiemap_extent_info *, u64 start,
		      u64 len);
	int (*update_time)(struct inode *, struct timespec *, int);
	int (*atomic_open)(struct inode *, struct dentry *,
			   struct file *, unsigned open_flag,
			   umode_t create_mode, int *opened);
	int (*tmpfile) (struct inode *, struct dentry *, umode_t);
	int (*set_acl)(struct inode *, struct posix_acl *, int);

	/* WARNING: probably going away soon, do not use! */
} ____cacheline_aligned;
\end{lstlisting}





\section{super\_block}

	内核使用该结构体来描述文件系统的功能和行为:

\begin{lstlisting}[language={[ANSI]C},
        basicstyle=\tiny\ttfamily,
        stringstyle=\color{purple},
        keywordstyle=\color{blue}\bfseries,
        commentstyle=\color{olive},
        directivestyle=\color{blue},
        frame=shadowbox,
        %framerule=0pt,
        %backgroundcolor=\color{pink},
        rulesepcolor=\color{red!20!green!20!blue!20}
        %rulesepcolor=\color{brown}
        %xleftmargin=2em,xrightmargin=2em,aboveskip=0em
        ]
        
struct super_block {
	struct list_head	s_list;		/* 指向super_block的链表 */
	dev_t			s_dev;		/* 设备标识符 */
	unsigned char		s_blocksize_bits;/* 以位为单位的块大小 */
	unsigned long		s_blocksize;/* 以字节为单位的块大小 */
	loff_t			s_maxbytes;	/* Max file size */
	struct file_system_type	*s_type;/* Filesystem type */
	const struct super_operations	*s_op;/*超级块方法*/
	const struct dquot_operations	*dq_op;/*磁盘限额方法 */
	const struct quotactl_ops	*s_qcop;/* 限额控制方法 */
	const struct export_operations *s_export_op;/* 导出方法 */
	unsigned long		s_flags;/* 挂载标志 */
	unsigned long		s_iflags;	/* internal SB_I_* flags */
	unsigned long		s_magic;/* 文件系统魔数 */
	struct dentry		*s_root;/* 目录挂载点 */
	struct rw_semaphore	s_umount;/* 卸载信号量 */
	int			s_count;/* 超级块引用计数 */
	atomic_t		s_active;/* 活动引用计数 */
#ifdef CONFIG_SECURITY
	void                    *s_security;/* 安全模块 */
#endif
	const struct xattr_handler **s_xattr;/*扩展的属性操作*/

	struct hlist_bl_head	s_anon;		/* anonymous dentries for (nfs) exporting */
	struct list_head	s_mounts;	/* list of mounts; _not_ for fs use */
	struct block_device	*s_bdev;/*相关的块设备*/
	struct backing_dev_info *s_bdi;
	struct mtd_info		*s_mtd;
	struct hlist_node	s_instances;
	unsigned int		s_quota_types;	/* Bitmask of supported quota types */
	struct quota_info	s_dquot;	/* Diskquota specific options */

	struct sb_writers	s_writers;

	char s_id[32];				/* Informational name */
	u8 s_uuid[16];				/* UUID */

	void 			*s_fs_info;	/* Filesystem private info */
	unsigned int		s_max_links;
	fmode_t			s_mode;

	/* Granularity of c/m/atime in ns.
	   Cannot be worse than a second */
	u32		   s_time_gran;

	/*
	 * The next field is for VFS *only*. No filesystems have any business
	 * even looking at it. You had been warned.
	 */
	struct mutex s_vfs_rename_mutex;	/* Kludge */

	/*
	 * Filesystem subtype.  If non-empty the filesystem type field
	 * in /proc/mounts will be "type.subtype"
	 */
	char *s_subtype;

	/*
	 * Saved mount options for lazy filesystems using
	 * generic_show_options()
	 */
	char __rcu *s_options;
	const struct dentry_operations *s_d_op; /* default d_op for dentries */

	/*
	 * Saved pool identifier for cleancache (-1 means none)
	 */
	int cleancache_poolid;

	struct shrinker s_shrink;	/* per-sb shrinker handle */

	/* Number of inodes with nlink == 0 but still referenced */
	atomic_long_t s_remove_count;

	/* Being remounted read-only */
	int s_readonly_remount;

	/* AIO completions deferred from interrupt context */
	struct workqueue_struct *s_dio_done_wq;
	struct hlist_head s_pins;

	/*
	 * Keep the lru lists last in the structure so they always sit on their
	 * own individual cachelines.
	 */
	struct list_lru		s_dentry_lru ____cacheline_aligned_in_smp;
	struct list_lru		s_inode_lru ____cacheline_aligned_in_smp;
	struct rcu_head		rcu;
	struct work_struct	destroy_work;

	struct mutex		s_sync_lock;	/* sync serialisation lock */

	/*
	 * Indicates how deep in a filesystem stack this SB is
	 */
	int s_stack_depth;

	/* s_inode_list_lock protects s_inodes */
	spinlock_t		s_inode_list_lock ____cacheline_aligned_in_smp;
	struct list_head	s_inodes;	/* all inodes */
};
        
\end{lstlisting}


\section{dentry}

\begin{lstlisting}[language={[ANSI]C},
        basicstyle=\tiny\ttfamily,
        stringstyle=\color{purple},
        keywordstyle=\color{blue}\bfseries,
        commentstyle=\color{olive},
        directivestyle=\color{blue},
        frame=shadowbox,
        %framerule=0pt,
        %backgroundcolor=\color{pink},
        rulesepcolor=\color{red!20!green!20!blue!20}
        %rulesepcolor=\color{brown}
        %xleftmargin=2em,xrightmargin=2em,aboveskip=0em
        ]

struct dentry {
	/* RCU lookup touched fields */
	unsigned int d_flags;		/* protected by d_lock */
	seqcount_t d_seq;		/* per dentry seqlock */
	struct hlist_bl_node d_hash;	/* lookup hash list */
	struct dentry *d_parent;	/* parent directory */
	struct qstr d_name;
	struct inode *d_inode;		/* Where the name belongs to - NULL is
					 * negative */
	unsigned char d_iname[DNAME_INLINE_LEN];	/* small names */

	/* Ref lookup also touches following */
	struct lockref d_lockref;	/* per-dentry lock and refcount */
	const struct dentry_operations *d_op;
	struct super_block *d_sb;	/* The root of the dentry tree */
	unsigned long d_time;		/* used by d_revalidate */
	void *d_fsdata;			/* fs-specific data */

	struct list_head d_lru;		/* LRU list */
	struct list_head d_child;	/* child of parent list */
	struct list_head d_subdirs;	/* our children */
	/*
	 * d_alias and d_rcu can share memory
	 */
	union {
		struct hlist_node d_alias;	/* inode alias list */
	 	struct rcu_head d_rcu;
	} d_u;
};


\end{lstlisting}
