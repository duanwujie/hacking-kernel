\chapter{文件系统}




\section{inode}

内核处理文件的关键是inode,每个文件(和目录)都有且只有一个对应的inode,其中包含元数据(如访问权限,上次
修改的日期,等等)和指向文件数据的指针。

\begin{lstlisting}[language={[ANSI]C},
        basicstyle=\tiny\ttfamily,
        stringstyle=\color{purple},
        keywordstyle=\color{blue}\bfseries,
        commentstyle=\color{olive},
        directivestyle=\color{blue},
        frame=shadowbox,
        %framerule=0pt,
        %backgroundcolor=\color{pink},
        rulesepcolor=\color{red!20!green!20!blue!20}
        %rulesepcolor=\color{brown}
        %xleftmargin=0em,xrightmargin=0em,aboveskip=0em
        ]


/*
 * Keep mostly read-only and often accessed (especially for
 * the RCU path lookup and 'stat' data) fields at the beginning
 * of the 'struct inode'
 */
struct inode {/* fs.h */
	umode_t			i_mode;/* 文件访问权限和所有权 */
	unsigned short		i_opflags;
	kuid_t			i_uid;/* uid about the file */
	kgid_t			i_gid;/* gid about the file */
	unsigned int		i_flags;

#ifdef CONFIG_FS_POSIX_ACL
	struct posix_acl	*i_acl;
	struct posix_acl	*i_default_acl;
#endif
	/* 负责管理结构性操作(如删除一个文件)和文件相关的元数据(例如属性) */
	const struct inode_operations	*i_op;
	struct super_block	*i_sb;
	struct address_space	*i_mapping;

#ifdef CONFIG_SECURITY
	void			*i_security;
#endif

	/* Stat data, not accessed from path walking */
	/* 对给定的文件系统,唯一的编号标识 */
	unsigned long		i_ino;
	/*
	 * Filesystems may only read i_nlink directly.  They shall use the
	 * following functions for modification:
	 *
	 *    (set|clear|inc|drop)_nlink
	 *    inode_(inc|dec)_link_count
	 */
	union {
		/* 记录使用该 inode 的硬链接总数 */
		const unsigned int i_nlink;
		unsigned int __i_nlink;
	};
	dev_t			i_rdev;
	
	loff_t			i_size;/* 文件大小 */
	struct timespec		i_atime;/* 最后访问时间 */
	struct timespec		i_mtime;/* 最后修改时间*/
	struct timespec		i_ctime;/* inode 最后修改时间 */
	spinlock_t		i_lock;	/* i_blocks, i_bytes, maybe i_size */
	unsigned short          i_bytes;
	unsigned int		i_blkbits;
	blkcnt_t		i_blocks;/*指定了按块存放的长度*/

#ifdef __NEED_I_SIZE_ORDERED
	seqcount_t		i_size_seqcount;
#endif

	/* Misc */
	unsigned long		i_state;
	struct mutex		i_mutex;

	unsigned long		dirtied_when;	/* jiffies of first dirtying */
	unsigned long		dirtied_time_when;

	struct hlist_node	i_hash;
	struct list_head	i_io_list;	/* backing dev IO list */
#ifdef CONFIG_CGROUP_WRITEBACK
	struct bdi_writeback	*i_wb;		/* the associated cgroup wb */

	/* foreign inode detection, see wbc_detach_inode() */
	int			i_wb_frn_winner;
	u16			i_wb_frn_avg_time;
	u16			i_wb_frn_history;
#endif
	struct list_head	i_lru;		/* inode LRU list */
	struct list_head	i_sb_list;
	union {
		struct hlist_head	i_dentry;
		struct rcu_head		i_rcu;
	};
	u64			i_version;
	atomic_t		i_count;/* 访问该inode的进程数目 */
	atomic_t		i_dio_count;
	atomic_t		i_writecount;
#ifdef CONFIG_IMA
	atomic_t		i_readcount; /* struct files open RO */
#endif

	const struct file_operations	*i_fop;	/* 用于操作文件中包含的数据 */
	struct file_lock_context	*i_flctx;
	struct address_space	i_data;
	struct list_head	i_devices;
	union {
		struct pipe_inode_info	*i_pipe;
		struct block_device	*i_bdev;
		struct cdev		*i_cdev;
		char			*i_link;
	};

	__u32			i_generation;

#ifdef CONFIG_FSNOTIFY
	__u32			i_fsnotify_mask; /* all events this inode cares about */
	struct hlist_head	i_fsnotify_marks;
#endif

	void			*i_private; /* fs or device private pointer */
};


\end{lstlisting}


\section{inode\_operations}

大多数请况下,各个函数指针成员的意义可以根据其名称推断。它们与对应的系统调用和用户空间工具
在名称方面非常相似。

\begin{lstlisting}[language={[ANSI]C},
        basicstyle=\tiny\ttfamily,
        stringstyle=\color{purple},
        keywordstyle=\color{blue}\bfseries,
        commentstyle=\color{olive},
        directivestyle=\color{blue},
        frame=shadowbox,
        %framerule=0pt,
        %backgroundcolor=\color{pink},
        rulesepcolor=\color{red!20!green!20!blue!20}
        %rulesepcolor=\color{brown}
        %xleftmargin=2em,xrightmargin=2em,aboveskip=0em
        ]
struct inode_operations {
	/* lookup 根据文件系统对象的名称(表示为字符串)查找其 inode 实例*/
	struct dentry * (*lookup) (struct inode *,struct dentry *, unsigned int);
	const char * (*follow_link) (struct dentry *, void **);
	int (*permission) (struct inode *, int);
	struct posix_acl * (*get_acl)(struct inode *, int);

	int (*readlink) (struct dentry *, char __user *,int);
	void (*put_link) (struct inode *, void *);

	int (*create) (struct inode *,struct dentry *, umode_t, bool);
	int (*link) (struct dentry *,struct inode *,struct dentry *);
	int (*unlink) (struct inode *,struct dentry *);
	int (*symlink) (struct inode *,struct dentry *,const char *);
	int (*mkdir) (struct inode *,struct dentry *,umode_t);
	int (*rmdir) (struct inode *,struct dentry *);
	int (*mknod) (struct inode *,struct dentry *,umode_t,dev_t);
	int (*rename) (struct inode *, struct dentry *,
			struct inode *, struct dentry *);
	int (*rename2) (struct inode *, struct dentry *,
			struct inode *, struct dentry *, unsigned int);
	int (*setattr) (struct dentry *, struct iattr *);
	int (*getattr) (struct vfsmount *mnt, struct dentry *, struct kstat *);
	int (*setxattr) (struct dentry *, const char *,const void *,size_t,int);
	ssize_t (*getxattr) (struct dentry *, const char *, void *, size_t);
	ssize_t (*listxattr) (struct dentry *, char *, size_t);
	int (*removexattr) (struct dentry *, const char *);
	int (*fiemap)(struct inode *, struct fiemap_extent_info *, u64 start,
		      u64 len);
	int (*update_time)(struct inode *, struct timespec *, int);
	int (*atomic_open)(struct inode *, struct dentry *,
			   struct file *, unsigned open_flag,
			   umode_t create_mode, int *opened);
	int (*tmpfile) (struct inode *, struct dentry *, umode_t);
	int (*set_acl)(struct inode *, struct posix_acl *, int);

	/* WARNING: probably going away soon, do not use! */
} ____cacheline_aligned;
\end{lstlisting}