\chapter{技术分析}

最後再次告诫投资者∶不要盲目跟风,不要做趋势尾部的“博傻者”,如果你真的希望押注欧元长期趋势向下,请耐心等待汇价充分修正後再进场,否则可能陷入过度贪婪的陷阱。

合适的入场价格很重要!

\section{布林带}
\begin{itemize}

\item 用来衡量市场的波动。
\item 他们的行为就像是迷你的支持和阻力水平。
\item 布林线反弹:一个策略上的概念,价格往往总是回到布林线中间地带。
\item 布林线挤压:当布林线挤压时,意味着市场是非常安静的,需要一个杰出的突破。 一旦爆发,无论在什么方面,当我们进入交易时都是在寻求突破。
\end{itemize}


\section{RSI}

\section{MACD}

\section{多重时间段分析}


多重时间段分析是透过观察相同货币对多个不同时间段图表而分析货币对的方法。其优点是藉观察较长的时间段,然后再看较短的时间段,继而再看更短的时间段,交易者将可对货币对如何移动有更详细的了解,从而在更有根据的情况下建立交易。


一般而言,交易者会选择三个时间段,而有关时间段将会根据交易者的个人交易策略而定。较长线交易者可以选择每周、每日及4小时图。较短线交易者可以选择4小时、1小时及15分钟图。重点是以当中最长线的图表来厘定“整体趋势”及应朝哪一个方向买卖。其后使用较短的时间段来“微调”在该个方向哪一个水平建立持仓。


您可能曾经听说过“趋势中存在趋势”。例如,在日图中,趋势可能是一个升势,而在4小时图中,则可能是跌势,在1小时图中可能是持平… 所有趋势都是关于同一个货币对。

在此情景下,整体趋势以日图为基础 – 即上升。然而,在这个升势之内,于4小时的时间段内则出现转势。转势很可能在某一点结束,而4小时时间段则与日图趋于一致。同样道理,在4小时时间段内存在1小时的趋势。随着1小时趋势与4小时趋于一致,而4小时与日图趋于一致,一个有机会更高的入市点将会出现。

总括来说,当最短的时间段完成转势(转势指我们在日图中注意到与趋势相反的走势),及开始重新朝每日趋势的方向移动时,我们便希望建立交易。这就是我们的入市信号。

试想像它们是一个密码锁的制栓,全部都有次序地趋于一致。

透过使用数个时间段,交易者可以在三个不同层面上深入了解货币对,及学习运用该些资料,以成功地在时间段所示成功机率最高的时候建立交易。

