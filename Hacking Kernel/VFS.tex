\chapter{VFS}



\section{sys\_mount流程}


\begin{code}
//linux-4.3.3/fs/namespace.c

/**
 * sys_mount - 挂载文件系统
 * @dev_name :设备名称`
 * @dir_name :挂载点路径
 * @type :文件系统类型
 * @flags :标志位供 do_mount 调用。
 * @data :选项信息
 */
SYSCALL_DEFINE5(mount, char __user *, dev_name, char __user *, dir_name,
		char __user *, type, unsigned long, flags, void __user *, data);
\end{code}

\begin{itemize}
\item 将用户态参数拷贝至内核态
\item 调用do\_mount完成主要挂载工作
\end{itemize}

\section{do\_mount流程}


其中get\_fs\_type:用于判断使用那个file\_system\_type

\section{注册一个文件系统}




\begin{lstlisting}[language={[ANSI]C},
        basicstyle=\tiny\ttfamily,
        stringstyle=\color{purple},
        keywordstyle=\color{blue}\bfseries,
        commentstyle=\color{olive},
        directivestyle=\color{blue},
        frame=shadowbox,
        %framerule=0pt,
        %backgroundcolor=\color{pink},
        rulesepcolor=\color{red!20!green!20!blue!20}
        %rulesepcolor=\color{brown}
        %xleftmargin=2em,xrightmargin=2em,aboveskip=0em
        ]
#include <linux/fs.h>
extern int register_filesystem(struct file_system_type *);
extern int unregister_filesystem(struct file_system_type *);
\end{lstlisting}


file\_systems:文件系统链表,后续对VFS的操作将围绕该链表展开。

\begin{itemize}
\item register\_filesystem:通过文件系统的名字在file\_systems链表中查找对应的文件系统,没有找到,则将文件新的文件系统加入链表。
\item unregister\_filesystem:将文件系统从file\_systems链表中删除
\item /proc/filesystems下显示了所有已经注册的文件系统。
\end{itemize}





\section{file\_system\_type}
	file\_system\_type用于描述一个文件系统:

\begin{lstlisting}[language={[ANSI]C},
        basicstyle=\tiny\ttfamily,
        stringstyle=\color{purple},
        keywordstyle=\color{blue}\bfseries,
        commentstyle=\color{olive},
        directivestyle=\color{blue},
        frame=shadowbox,
        %framerule=0pt,
        %backgroundcolor=\color{pink},
        rulesepcolor=\color{red!20!green!20!blue!20}
        %rulesepcolor=\color{brown}
        %xleftmargin=2em,xrightmargin=2em,aboveskip=0em
        ]
        
struct file_system_type {
	const char *name;/* 文件系统的名字 */
	int fs_flags;	/* 文件系统类型标志 */
	struct dentry *(*mount) (struct file_system_type *, int,
		       const char *, void *);/* 当挂在一个文件系统时调用*/
	void (*kill_sb) (struct super_block *);/*  */
	struct module *owner;/* 文件系统模块,初始化为THIS_MODULE */
	struct file_system_type * next;/*VFS内部使用,初始化为空*/
	struct hlist_head fs_supers;/* supperblock list */
	/* 相关锁 */
	/* ... */
};
        
        
\end{lstlisting}


 The mount() method must return the root dentry of the tree requested by
 caller.  An active reference to its superblock must be grabbed and the
 superblock must be locked.  On failure it should return ERR\_PTR(error).
 
 The arguments match those of mount(2) and their interpretation
 depends on filesystem type.  E.g. for block filesystems, dev\_name is
 interpreted as block device name, that device is opened and if it
 contains a suitable filesystem image the method creates and initializes
 struct super\_block accordingly, returning its root dentry to caller.
 
 ->mount() may choose to return a subtree of existing filesystem - it
 doesn't have to create a new one.  The main result from the caller's
 point of view is a reference to dentry at the root of (sub)tree to
 be attached; creation of new superblock is a common side effect.
 
 The most interesting member of the superblock structure that the
 mount() method fills in is the "s\_op" field. This is a pointer to
 a "struct super\_operations" which describes the next level of the
 filesystem implementation.
 
 Usually, a filesystem uses one of the generic mount() implementations
 and provides a fill\_super() callback instead. The generic variants are:

	\begin{itemize}
	\item   mount\_bdev: mount a filesystem residing on a block device
	\item   mount\_nodev: mount a filesystem that is not backed by a device
	\item   mount\_single: mount a filesystem which shares the instance between all mounts
	\end{itemize}
 A fill\_super() callback implementation has the following arguments:
 	\begin{itemize}
   \item struct super\_block *sb: the superblock structure. The callback
         must initialize this properly.
 
    \item void *data: arbitrary mount options, usually comes as an ASCII
         string (see "Mount Options" section)
 
    \item int silent: whether or not to be silent on error
	\end{itemize}
VFS使用面向对象的设计思路,VFS中有4个主要的对象类型:

\begin{itemize}
\item 超级块对象(super\_block):它表示一个具体的已安装的文件系统
\item 索引节点对象(inode):它表示一个具体的文件
\item 目录项对象(dentry):它表示一个目录项,是路径的一个组成部分。
\item 文件对象(file):它表示进程打开的文件。
\end{itemize}

VFS将目录当作文件来处理,所以不存在目录对象,目录项代表的是路径中的一个组成部分。


\section{super\_block}

A superblock object represents a mounted filesystem.

\begin{code}
        
struct super_block {
	struct list_head	s_list;		/* 指向super_block的链表 */
	dev_t			s_dev;		/* 设备标识符 */
	unsigned char		s_blocksize_bits;/* 以位为单位的块大小 */
	unsigned long		s_blocksize;/* 以字节为单位的块大小 */
	loff_t			s_maxbytes;	/* Max file size */
	struct file_system_type	*s_type;/* Filesystem type */
	const struct super_operations	*s_op;/*超级块方法*/
	const struct dquot_operations	*dq_op;/*磁盘限额方法 */
	const struct quotactl_ops	*s_qcop;/* 限额控制方法 */
	const struct export_operations *s_export_op;/* 导出方法 */
	unsigned long		s_flags;/* 挂载标志 */
	unsigned long		s_iflags;	/* internal SB_I_* flags */
	unsigned long		s_magic;/* 文件系统魔数 */
	struct dentry		*s_root;/* 目录挂载点 */
	struct rw_semaphore	s_umount;/* 卸载信号量 */
	int			s_count;/* 超级块引用计数 */
	atomic_t		s_active;/* 活动引用计数 */
#ifdef CONFIG_SECURITY
	void                    *s_security;/* 安全模块 */
#endif
	const struct xattr_handler **s_xattr;/*扩展的属性操作*/

	struct hlist_bl_head	s_anon;		/* anonymous dentries for (nfs) exporting */
	struct list_head	s_mounts;	/* list of mounts; _not_ for fs use */
	struct block_device	*s_bdev;/*相关的块设备*/
	struct backing_dev_info *s_bdi;
	struct mtd_info		*s_mtd;
	struct hlist_node	s_instances;
	unsigned int		s_quota_types;	/* Bitmask of supported quota types */
	struct quota_info	s_dquot;	/* Diskquota specific options */

	struct sb_writers	s_writers;

	char s_id[32];				/* Informational name */
	u8 s_uuid[16];				/* UUID */

	void 			*s_fs_info;	/* 文件系统私有数据 */
	unsigned int		s_max_links;
	fmode_t			s_mode;

	/* Granularity of c/m/atime in ns.
	   Cannot be worse than a second */
	u32		   s_time_gran;

	/*
	 * The next field is for VFS *only*. No filesystems have any business
	 * even looking at it. You had been warned.
	 */
	struct mutex s_vfs_rename_mutex;	/* Kludge */

	/*
	 * Filesystem subtype.  If non-empty the filesystem type field
	 * in /proc/mounts will be "type.subtype"
	 */
	char *s_subtype;

	/*
	 * Saved mount options for lazy filesystems using
	 * generic_show_options()
	 */
	char __rcu *s_options;
	const struct dentry_operations *s_d_op; /* default d_op for dentries */

	/*
	 * Saved pool identifier for cleancache (-1 means none)
	 */
	int cleancache_poolid;

	struct shrinker s_shrink;	/* per-sb shrinker handle */

	/* Number of inodes with nlink == 0 but still referenced */
	atomic_long_t s_remove_count;

	/* Being remounted read-only */
	int s_readonly_remount;

	/* AIO completions deferred from interrupt context */
	struct workqueue_struct *s_dio_done_wq;
	struct hlist_head s_pins;

	/*
	 * Keep the lru lists last in the structure so they always sit on their
	 * own individual cachelines.
	 */
	struct list_lru		s_dentry_lru ____cacheline_aligned_in_smp;
	struct list_lru		s_inode_lru ____cacheline_aligned_in_smp;
	struct rcu_head		rcu;
	struct work_struct	destroy_work;

	struct mutex		s_sync_lock;	/* sync serialisation lock */

	/*
	 * Indicates how deep in a filesystem stack this SB is
	 */
	int s_stack_depth;

	/* s_inode_list_lock protects s_inodes */
	spinlock_t		s_inode_list_lock ____cacheline_aligned_in_smp;
	struct list_head	s_inodes;	/* all inodes */
};
        
\end{code}



\section{inode}

内核处理文件的关键是inode,每个文件(和目录)都有且只有一个对应的inode,其中包含元数据(如访问权限,上次
修改的日期,等等)和指向文件数据的指针。

\begin{lstlisting}[language={[ANSI]C},
        basicstyle=\tiny\ttfamily,
        stringstyle=\color{purple},
        keywordstyle=\color{blue}\bfseries,
        commentstyle=\color{olive},
        directivestyle=\color{blue},
        frame=shadowbox,
        %framerule=0pt,
        %backgroundcolor=\color{pink},
        rulesepcolor=\color{red!20!green!20!blue!20}
        %rulesepcolor=\color{brown}
        %xleftmargin=0em,xrightmargin=0em,aboveskip=0em
        ]


/*
 * Keep mostly read-only and often accessed (especially for
 * the RCU path lookup and 'stat' data) fields at the beginning
 * of the 'struct inode'
 */
struct inode {/* fs.h */
	umode_t			i_mode;/* 文件访问权限和所有权 */
	unsigned short		i_opflags;
	kuid_t			i_uid;/* uid about the file */
	kgid_t			i_gid;/* gid about the file */
	unsigned int		i_flags;

#ifdef CONFIG_FS_POSIX_ACL
	struct posix_acl	*i_acl;
	struct posix_acl	*i_default_acl;
#endif
	/* 负责管理结构性操作(如删除一个文件)和文件相关的元数据(例如属性) */
	const struct inode_operations	*i_op;
	struct super_block	*i_sb;
	struct address_space	*i_mapping;

#ifdef CONFIG_SECURITY
	void			*i_security;
#endif

	/* Stat data, not accessed from path walking */
	/* 对给定的文件系统,唯一的编号标识 */
	unsigned long		i_ino;
	/*
	 * Filesystems may only read i_nlink directly.  They shall use the
	 * following functions for modification:
	 *
	 *    (set|clear|inc|drop)_nlink
	 *    inode_(inc|dec)_link_count
	 */
	union {
		/* 记录使用该 inode 的硬链接总数 */
		const unsigned int i_nlink;
		unsigned int __i_nlink;
	};
	dev_t			i_rdev;
	
	loff_t			i_size;/* 文件大小 */
	struct timespec		i_atime;/* 最后访问时间 */
	struct timespec		i_mtime;/* 最后修改时间*/
	struct timespec		i_ctime;/* inode 最后修改时间 */
	spinlock_t		i_lock;	/* i_blocks, i_bytes, maybe i_size */
	unsigned short          i_bytes;
	unsigned int		i_blkbits;
	blkcnt_t		i_blocks;/*指定了按块存放的长度*/

#ifdef __NEED_I_SIZE_ORDERED
	seqcount_t		i_size_seqcount;
#endif

	/* Misc */
	unsigned long		i_state;
	struct mutex		i_mutex;

	unsigned long		dirtied_when;	/* jiffies of first dirtying */
	unsigned long		dirtied_time_when;

	struct hlist_node	i_hash;
	struct list_head	i_io_list;	/* backing dev IO list */
#ifdef CONFIG_CGROUP_WRITEBACK
	struct bdi_writeback	*i_wb;		/* the associated cgroup wb */

	/* foreign inode detection, see wbc_detach_inode() */
	int			i_wb_frn_winner;
	u16			i_wb_frn_avg_time;
	u16			i_wb_frn_history;
#endif
	struct list_head	i_lru;		/* inode LRU list */
	struct list_head	i_sb_list;
	union {
		struct hlist_head	i_dentry;
		struct rcu_head		i_rcu;
	};
	u64			i_version;
	atomic_t		i_count;/* 访问该inode的进程数目 */
	atomic_t		i_dio_count;
	atomic_t		i_writecount;
#ifdef CONFIG_IMA
	atomic_t		i_readcount; /* struct files open RO */
#endif

	const struct file_operations	*i_fop;	/* 用于操作文件中包含的数据 */
	struct file_lock_context	*i_flctx;
	struct address_space	i_data;
	struct list_head	i_devices;
	union {
		struct pipe_inode_info	*i_pipe;
		struct block_device	*i_bdev;
		struct cdev		*i_cdev;
		char			*i_link;
	};

	__u32			i_generation;

#ifdef CONFIG_FSNOTIFY
	__u32			i_fsnotify_mask; /* all events this inode cares about */
	struct hlist_head	i_fsnotify_marks;
#endif

	void			*i_private; /* fs or device private pointer */
};


\end{lstlisting}


\section{inode\_operations}

大多数请况下,各个函数指针成员的意义可以根据其名称推断。它们与对应的系统调用和用户空间工具
在名称方面非常相似。

\begin{lstlisting}[language={[ANSI]C},
        basicstyle=\tiny\ttfamily,
        stringstyle=\color{purple},
        keywordstyle=\color{blue}\bfseries,
        commentstyle=\color{olive},
        directivestyle=\color{blue},
        frame=shadowbox,
        %framerule=0pt,
        %backgroundcolor=\color{pink},
        rulesepcolor=\color{red!20!green!20!blue!20}
        %rulesepcolor=\color{brown}
        %xleftmargin=2em,xrightmargin=2em,aboveskip=0em
        ]
struct inode_operations {
	/* lookup 根据文件系统对象的名称(表示为字符串)查找其 inode 实例*/
	struct dentry * (*lookup) (struct inode *,struct dentry *, unsigned int);
	const char * (*follow_link) (struct dentry *, void **);
	int (*permission) (struct inode *, int);
	struct posix_acl * (*get_acl)(struct inode *, int);

	int (*readlink) (struct dentry *, char __user *,int);
	void (*put_link) (struct inode *, void *);

	int (*create) (struct inode *,struct dentry *, umode_t, bool);
	int (*link) (struct dentry *,struct inode *,struct dentry *);
	int (*unlink) (struct inode *,struct dentry *);
	int (*symlink) (struct inode *,struct dentry *,const char *);
	int (*mkdir) (struct inode *,struct dentry *,umode_t);
	int (*rmdir) (struct inode *,struct dentry *);
	int (*mknod) (struct inode *,struct dentry *,umode_t,dev_t);
	int (*rename) (struct inode *, struct dentry *,
			struct inode *, struct dentry *);
	int (*rename2) (struct inode *, struct dentry *,
			struct inode *, struct dentry *, unsigned int);
	int (*setattr) (struct dentry *, struct iattr *);
	int (*getattr) (struct vfsmount *mnt, struct dentry *, struct kstat *);
	int (*setxattr) (struct dentry *, const char *,const void *,size_t,int);
	ssize_t (*getxattr) (struct dentry *, const char *, void *, size_t);
	ssize_t (*listxattr) (struct dentry *, char *, size_t);
	int (*removexattr) (struct dentry *, const char *);
	int (*fiemap)(struct inode *, struct fiemap_extent_info *, u64 start,
		      u64 len);
	int (*update_time)(struct inode *, struct timespec *, int);
	int (*atomic_open)(struct inode *, struct dentry *,
			   struct file *, unsigned open_flag,
			   umode_t create_mode, int *opened);
	int (*tmpfile) (struct inode *, struct dentry *, umode_t);
	int (*set_acl)(struct inode *, struct posix_acl *, int);

	/* WARNING: probably going away soon, do not use! */
} ____cacheline_aligned;
\end{lstlisting}








\section{dentry}

\begin{lstlisting}[language={[ANSI]C},
        basicstyle=\tiny\ttfamily,
        stringstyle=\color{purple},
        keywordstyle=\color{blue}\bfseries,
        commentstyle=\color{olive},
        directivestyle=\color{blue},
        frame=shadowbox,
        %framerule=0pt,
        %backgroundcolor=\color{pink},
        rulesepcolor=\color{red!20!green!20!blue!20}
        %rulesepcolor=\color{brown}
        %xleftmargin=2em,xrightmargin=2em,aboveskip=0em
        ]

struct dentry {
	/* RCU lookup touched fields */
	unsigned int d_flags;		/* protected by d_lock */
	seqcount_t d_seq;		/* per dentry seqlock */
	struct hlist_bl_node d_hash;	/* lookup hash list */
	struct dentry *d_parent;	/* parent directory */
	struct qstr d_name;
	struct inode *d_inode;		/* Where the name belongs to - NULL is
					 * negative */
	unsigned char d_iname[DNAME_INLINE_LEN];	/* small names */

	/* Ref lookup also touches following */
	struct lockref d_lockref;	/* per-dentry lock and refcount */
	const struct dentry_operations *d_op;
	struct super_block *d_sb;	/* The root of the dentry tree */
	unsigned long d_time;		/* used by d_revalidate */
	void *d_fsdata;			/* fs-specific data */

	struct list_head d_lru;		/* LRU list */
	struct list_head d_child;	/* child of parent list */
	struct list_head d_subdirs;	/* our children */
	/*
	 * d_alias and d_rcu can share memory
	 */
	union {
		struct hlist_node d_alias;	/* inode alias list */
	 	struct rcu_head d_rcu;
	} d_u;
};


\end{lstlisting}
