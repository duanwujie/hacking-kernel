\chapter{经济数据分析}

\section{CRB指数}

反映的是美国商品价格的总体波动,能使机构和个人投资者利用指数交易而获得商品价格综合变动带来的获利机会。


在分析数据时应当从数据本身进行分析,即从原数据分析,不能使用别人消化过后的观点!

汇率是由外汇的供求关系决定的,以下是影响汇率的主要因素:
\begin{itemize}
\item 利率
\item 经济活动水平
\item 非投机资本流
\item 投机资本流
\item 贸易平衡
\item 政府预算
\item 地缘政治事件
\end{itemize}

\section{利率}

\begin{itemize}
\item \textbf{存款利率(Deposit interest rate)}: 个人或企业将钱存入银行,银行所需付出的货币价格!
\item \textbf{借款利率(Lending interest rate )}:银行将钱借给个人或企业,个人或企业所需付出的货币价格!
\item \textbf{真实利率(Real interest rate)}:去除通货膨胀后的利率。
\end{itemize}

\section{经济活动水平}
\section{非投机资本流}
\section{投机资本流}
\section{贸易平衡}
\section{政府预算}


\section{美国经济数据分析}



\subsection{利率}



贷款利率从08年金融危机以来从5.1跌至3.3并连续4年内保持在3.3,降低借款利率的原因是美联储想

\href{http://data.worldbank.org/indicator/FR.INR.RINR/countries/US?display=graph}{美国真实利率},从2011年到现在来看,美国的真实利率在逐步增加,这样今天调高利率的可能性也可能会增大



\subsection{通货膨胀率}

通货膨胀的相关数据一般反应了总体物价水平的上升速度。通货膨胀会造成购买力下降,所以调控通货膨胀非常重要。通常来说通胀率上升标志着该经济发展的过快,而通胀率下降标志着经济急剧萎缩。

\begin{itemize}
\item 消费者物价指数(CPI)
\item 通货膨胀率计算公式
\end{itemize}
$$Inflation Rate=\frac{Current CPI-Base CPI}{Base CPI}*100\%$$

\subsection{就业率}
在一个经济体中就业率越高说明经济状况越良好这个指标和利率一样是个关键指标
\begin{itemize}
\item 非农就业人口,上个月新增的非农就业机会
\item 失业率:正在积极寻找工作而尚未获得工作职位的人所占的比例
\end{itemize}

就业人口从2008年的负值,就业人数开始逐步增加,这两年每个月的非农就业人口趋于稳定,说明美国经济在往好的方面发展!

失业率:从2008年~2009年增加,从2011年~2015年开始减少,说明美国经济在往好的方面发展!

\subsection{GDP国内生产总值}
\href{http://data.worldbank.org/indicator/NY.GDP.MKTP.KD.ZG/countries/1W-US?display=graph}{GDP}


GDP是在某一既定时期一个国家内生产的所有最终物品与劳务的市场价值,美国的GDP增长率是相对于2005年的GDP来说的!


经济增长的相关数据显示了经济体的经济发展程度,国家经济通常有3个发展方向:一是发展,二是停滞,三是收缩,经济体在不断发展时,该经济的各个方面都运作良好,并实现盈利性增长。如果经济体总是处于停滞或者收缩状态,其货币会遭受打击,走势疲软!疲软你懂的!

经济增长的的情况主要表现在国内生产总值中!


\subsection{国际贸易收支}

\subsection{海外投资}
\subsection{消费信心}



\section{加拿大经济数据分析}

\href{http://data.worldbank.org/indicator/FR.INR.RINR/countries/CA?display=graph}{加拿大真实利率}

\section{欧盟区经济数据分析}


\section{澳大利亚数据分析}
